\documentclass{article}
% ----- Preamble
\usepackage[utf8]{inputenc} % police encodee en latin1=iso8859-1=Windows Latin 1 %
\usepackage[french]{babel} % police fr %
\usepackage{hyperref} % pour les references %
\usepackage{amsmath} % pour les formules de maths %
\usepackage{amssymb} % pour les symboles maths %
\usepackage{amsthm} % pour la mise en forme des theoremes %
\usepackage{aeguill} % pour les guillemets et accents francais %
\usepackage{listings} % pour les listings de code %
\usepackage{helvet} % police helvetica %
\usepackage{graphicx}
\usepackage{centernot}
\usepackage{dsfont}
% modification des dimensions de la page et de son centrage %
\topmargin 0.0cm
\oddsidemargin 0.2cm
\textwidth 16cm 
\textheight 21cm
\footskip 0.0cm

\title{Traitement d'Image et du Signal - TP5}
\author{Laurent Cetinsoy, Karim Kouki, Aris Tritas }
\date{\today}

\begin{document}
\maketitle

\section{Egalisation d'histogramme}
Les libraires standard Octave et Matlab implémentent la fonction \textsf{histeq}. Ci-dessous un exemple d'histogramme saturé qui a été égalisé.

\begin{figure}[h]
	\includegraphics[width=0.5\textwidth]{hist_orig.png}
	\includegraphics[width=0.5\textwidth]{hist_eq.png}
  \caption{De gauche à droite $\sigma^2 = \frac{1}{100}, \frac{1}{10}, 1$}
\end{figure}

\section{Méthode dite du Midway}
Nous proposons d'utiliser une image en niveaux de gris, et une autre en couleur.

\begin{figure}[h]
	\includegraphics[width=0.33\textwidth]{NotreDame1.png}
	\includegraphics[width=0.33\textwidth]{NotreDame2.png}
	\includegraphics[width=0.33\textwidth]{NotreDame1M.png}
  \caption{De gauche à droite: Notre-Dame sur-exposée, sous-exposée, et midway}
\end{figure}

\begin{figure}[h]
	\includegraphics[width=0.33\textwidth]{P1c.jpg}
	\includegraphics[width=0.33\textwidth]{P2c.jpg}
	\includegraphics[width=0.33\textwidth]{P1Mc.png}
  \caption{De gauche à droite: patch sur-exposé, sous-exposé et midway}
\end{figure}

\section{Images de bruit gaussien}
Chacune des images ci-dessous ont des pixels distribués selon la loi normale: $I \sim \mathcal{N}(\frac{1}{2}, \sigma^2)$

\begin{figure}[h]
	\includegraphics[width=0.33\textwidth]{noise001.png}
	\includegraphics[width=0.33\textwidth]{noise01.png}
	\includegraphics[width=0.33\textwidth]{noise10.png}
  \caption{De gauche à droite $\sigma^2 = \frac{1}{100}, \frac{1}{10}, 1$}
\end{figure}



\section{Filtre bilatéral}
L'objectif est d'implémenter une forme simple du filtre bilatéral.
\subsection{Implémentation}
\subsubsection*{Q1}
$$w \in \mathbb{N}^+ \;, \; y \in \Omega(p) \rightarrow (y-p) \in {[{-w}, w]}^2$$
\subsubsection*{Q2}
Soit $\Omega(p) \ni y \rightarrow f_s(\parallel y-p \parallel_{\ell^2})$. 
D'après la question précédente: $ 0 \leq \parallel y-p \parallel_{\ell^2} \leq 2 w^2 $

Or $f_s$ est décroissante sur $\mathbb{R}_+$, donc: $$\exp(-(\frac{w}{\sigma_s})^2) \leq f_s(\parallel y-p \parallel_{\ell^2}) \leq 0 $$

\subsubsection*{Q3} D'après l'équation 3, nous avons besoin des points du sous-ensemble ci-dessous pour calculer $u_\text{denoised}(p)$ avec p fixé: 

$$\big \{ \: u(y_1, y_2) \;|\; y_1 \in {[p_1 - w, p_1 + w]}, \; y_2 \in {[p_2 - w, p_2 + w]}\: \big \}  $$

\textbf{Remarque: } Il s'agit d'une fenêtre carrée de taille fixe autour du pixel d'intérêt.
\subsubsection*{Q4}
Les conditions que les indices $p_1$ et $p_2$ doivent satisfaire sont:
 $$p_i + w < N \;\text{et}\; p_i - w > 0 \;,\; i \in \{1, 2\} \rightarrow p_1, p_2 \in {[w, N-w]}$$
\subsubsection*{Q5}
On pose les changements de variable: $i_1 = y_1 - p_1 + w + 1$ et $i_2 = y_2 - p_2 + w + 1$
Ainsi, par ré-indexation, on réécrit l'équation (3):
\begin{equation*}\begin{split}
u_\text{denoised}(p_1, p_2) = \frac{1}{C} \displaystyle\sum_{i_1 = 1}^{2w + 1}
\displaystyle\sum_{i_2 = 1}^{2w + 1} & u(i_1+p_1 - w - 1, i_2 + p_2 - w - 1) \\
& \exp{ \big ( - \frac{(i_1 - w - 1)^2 + (i_2 -w -1)^2 }{2\sigma_s^2} \big )} \\
& \exp{ \big ( - \frac{[u(i_1 + p_1 - w - 1, i_2 + p_2 - w - 1) - u(p_1, p_2)]^2}{2\sigma_i^2} \big )}
\end{split}\end{equation*}

\subsubsection*{Q6}
Ré-écrivons la formule précédente avec les fonctions $S$ et $\tilde{u}$ données:
\begin{equation*}\begin{split}
u_\text{denoised}(p_1, p_2) = \frac{1}{C} \displaystyle\sum_{i_1 = 1}^{2w + 1}
\displaystyle\sum_{i_2 = 1}^{2w + 1} & \tilde{u}(i_1, i_2) S(i_1,i_2) \exp{ \big ( - \frac{[\tilde{u}(i_1, i_2) - u(p_1, p_2)]^2}{2\sigma_i^2} \big )}
\end{split}\end{equation*}

\subsubsection*{Q7}
La constante de normalisation s'écrit de même:
\begin{equation*}\begin{split}
C = \displaystyle\sum_{i_1 = 1}^{2w + 1}
\displaystyle\sum_{i_2 = 1}^{2w + 1} S(i_1,i_2) \exp{ \big ( - \frac{[\tilde{u}(i_1, i_2) - u(p_1, p_2)]^2}{2\sigma_i^2} \big )}
\end{split}\end{equation*}
\subsubsection*{Q8}
D'après ce qui a été dit précédemment, trois matrices seront nécessaires pour implémenter le filtre:
\begin{itemize}
\item une matrice de la taille de l'image pour conserver les résultats de $u_denoised(p)$ durant le parcours de l'image
\item une matrice de taille $w \times w$ à valeurs constantes pour stocker les valeurs du kernel spatial
\item une matrice de taille $w \times w$ à valeurs variables pour stocker les valeurs du kernel d'intensité
\end{itemize}
\subsubsection*{Q9. Pseudo-code de l'algorithme}
\subsection{Analyse}
\subsubsection*{Q10. $\sigma_i \rightarrow + \infty $}
On tend à uniformiser sur les intensités voisines. L'on observe par exemple un ``aplatissement'' des tâches spectrales, c'est-à-dire une atténuation de celles-ci sur leur bord.
\subsubsection*{Q11. $\sigma_i \rightarrow 0 $}
Techniquement, le kernel spatial est seul pris en compte. En pratique, l'image originale reste inchangée.
\subsubsection*{Q12. $\sigma_s \rightarrow + \infty $}
On tend à uniformiser sur les valeurs des couleurs voisines (ce qui rappelle une convolution gaussienne).
\subsubsection*{Q13. $\sigma_s \rightarrow 0 $}
Le kernel d'intensité est seul pris en compte. De même, en pratique, l'image originale reste inchangée.

\subsection*{Comparaison de l'histogramme de l'image originale et de l'histogramme de l'image bruitée}

\section{Comparaison zoom zéro padding et TCD}
La transformée en cosinus discret implicite se fait en répliquant une image par symmétrie miroir, elle est ainsi quadruplée. Ci-dessous l'on compare cette méthode et un zéro padding simple. L'on constate bien que les raies produites par le zéro padding disparaissent par la périodisation implicite de la TCD.

\begin{figure}[h]
	\includegraphics[width=0.33\textwidth]{P1c.jpg}
	\includegraphics[width=0.33\textwidth]{P2c.jpg}
	\includegraphics[width=0.33\textwidth]{P1Mc.png}
  \caption{De gauche à droite: patch sur-exposé, sous-exposé et midway}
\end{figure}
\end{document}