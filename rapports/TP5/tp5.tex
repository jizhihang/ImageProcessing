\documentclass{article}
% ----- Preamble
\usepackage[utf8]{inputenc} % police encodee en latin1=iso8859-1=Windows Latin 1 %
\usepackage[french]{babel} % police fr %
\usepackage{hyperref} % pour les references %
\usepackage{amsmath} % pour les formules de maths %
\usepackage{amssymb} % pour les symboles maths %
\usepackage{amsthm} % pour la mise en forme des theoremes %
\usepackage{aeguill} % pour les guillemets et accents francais %
\usepackage{listings} % pour les listings de code %
\usepackage{helvet} % police helvetica %
\usepackage{graphicx}
\usepackage{centernot}
\usepackage{dsfont}
% modification des dimensions de la page et de son centrage %
\topmargin 0.0cm
\oddsidemargin 0.2cm
\textwidth 16cm 
\textheight 21cm
\footskip 0.0cm

\title{Traitement d'Image et du Signal - TP5}
\author{Laurent Cetinsoy, Karim Kouki, Aris Tritas }
\date{\today}

\begin{document}
\maketitle

\section{Egalisation d'histogramme}

\section{Méthode Midway}

\section{Images aléatoires}

On observe des images dont les pixel sont tous issus de réalisations  pour d'une même loi lois normales $P ~ \N(0.5, \sigma)$ avec $\sigma = 1/100, 1/10, 1$ respectivement. 




\section{Bilateral filter}


\section{Comparaison zoom zéro padding et TCD}

\begin{enumerate}
\item Appliquer l'égalisation d'histogramme à quelques images. On pourra
utiliser une commande Matlab (il faut la chercher!) pour l'égalisation
(elle fait tout).
\item Implémenter la méthode midway. On pourra prendre des paires d'images
que l'on a pris sois même et/ou celle dans l'archive "images.zip".

On pourra se restreindre au cas où les deux images ont même taille pour
profiter de l'implémentation avec les tris.

\item  Observer une image (e.g. 256x256) qui est la réalisation, en chaque
pixel, de variables aléatoires gaussiennes (centrées en 1/2) i.i.d pour
quelques écarts-types (1/100,1/10,1). Dynamique signal: [0,1].

\item Répondre aux questions du document bilateral\_filter\_2016.pdf et en
déduire l'implémentation du filtre bilatéral. Vous pourrez vous aider du
squelette dans ?

(On pourra en profiter pour visualiser l'histogramme de l'image
originale et le comparer à l'histogramme de l'image bruitée).

\item Comparer le zoom par zéro padding et le zoom par transformée en
cosinus discret implicite.
\end{enumerate}
\end{document}