\documentclass[12pt]{article}

% ----- Preamble
\usepackage[utf8]{inputenc} % police encodee en latin1=iso8859-1=Windows Latin 1 %
\usepackage[french]{babel} % police fr %
\usepackage{hyperref} % pour les references %
\usepackage{amsmath} % pour les formules de maths %
\usepackage{amssymb} % pour les symboles maths %
\usepackage{amsthm} % pour la mise en forme des theoremes %
\usepackage{aeguill} % pour les guillemets et accents francais %
\usepackage{listings} % pour les listings de code %
\usepackage{helvet} % police helvetica %
\usepackage{graphicx}
\usepackage{centernot}
\usepackage{dsfont}
% modification des dimensions de la page et de son centrage %
\topmargin 0.0cm
\oddsidemargin 0.2cm
\textwidth 16cm 
\textheight 21cm
\footskip 0.0cm

\title{Traitement d'Image et du Signal - TP3}
\author{Laurent Cetinsoy, Karim Kouki, Aris Tritas }
\date{\today}

\begin{document}
\maketitle

\begin{abstract}
L'objectif de ce TP est de calculer des convolutions, vérifier la validité de filtres linéaires, extraire leur réponse impulsionnelle. Par ailleus nous mettons en pratique la méthode de rotation par distortion de L.Yarolavsky.
\end{abstract}

\section*{Introduction}
Nous avons donné, lors du TP précédent, l'idée d'algorithme faisant une rotation efficace dans l'espace réel par succession de translations dans le domaine de Fourier. Par souçis de complétude, nous la ré-écrivons ci-suit.

\section*{Rotation d'image}
Pour chaque point $(x, y)$ de l'image originale nous souhaitons utiliser la TFD pour calculer le point $(x', y')$ résultant d'une rotation d'angle  $\theta \in {[0, 2\pi]}$. L'on peut définir la rotation par la matrice \textbf{M} ci-dessous:
$$\begin{pmatrix}
x' \\ y'
\end{pmatrix}=
\begin{pmatrix}
\cos \theta & -\sin \theta \\
\sin \theta & \cos \theta
\end{pmatrix}
\begin{pmatrix}
x \\ y
\end{pmatrix}=\textbf{M}
\begin{pmatrix}
x \\ y
\end{pmatrix}
$$
où \textbf{M} peut être ré-exprimée comme suit (e.g. \cite{paeth86}):
$$M =
\begin{pmatrix}
1 & -\tan \frac{\theta}{2} \\
0 & 1
\end{pmatrix}
\begin{pmatrix}
1 & 0 \\
\sin \theta & 1
\end{pmatrix}
\begin{pmatrix}
1 & -\tan \frac{\theta }{2}\\
0 & 1
\end{pmatrix}
$$
L'idée de \cite{unser95} \cite{larkin97} est d'utiliser trois convolutions linéaires (et donc séparables) qui distordent l'image successivement selon les axes $x$, $y$ et $x$. 
Chacune s'écrit comme une translation dans le domaine de Fourier. La distortion $u_{dx}(x, y) = u(x + ay, y)$ pour l'axe $x$ (où $a$ contrôle l'angle) s'exprime par l'opérateur suivant:
$$ D_x(\xi) = \mathcal{F}\{u_d(x, y)\} = \mathcal{F}\{u(x, y)\} e^{-2 i \pi \xi a y}$$
Le signal réel distordu sur $x$ par la transformée inverse : $ u_{dx}(x, y) = \mathcal{F}^{-1}\{D_x(\xi) \} $ \newline
Si l'on répête cette opération pour $y$ et encore une fois pour $x$ l'on retrouve le signal réel pivoté. Il s'agit bien de trois translations, donc un algorithme efficace qui les effectue peut être donné. A noter que cette méthode induit une perte d'information près des bords, il convient donc d'insérer l'image originale dans un cadre.

\section*{Exercices}

Notons le signal d'entrée $e(t)$ et le signal de sortie $s(t)$.


\textbf{Définition} Système linéaire: pour $\alpha_1, \alpha_2 \in \mathbb{R}$\newline
Si $e_1(t) \rightarrow s_1(t)$ et $e_2(t) \rightarrow s_2(t)$, alors $\alpha_1 e_1(t) + \alpha_2 e_2(t) \rightarrow \alpha_1 s_1(t) + \alpha_2 s_2(t)$

\textbf{Définition} Système invariant: pour $\tau \in \mathbb{R}$ \newline
Si $e(t) \rightarrow s(t)$ alors $e(t-\tau) \rightarrow s(t-\tau)$, $\tau$ étant une constante de décalalage.

\textbf{Définition} Produit de convolution de deux suites $u_n$ et $h_n$: \newline
$$ (u \otimes h)(n) = \sum_{k \in \mathbb{Z}} u_{n-k}h_k $$

\textbf{Définition} Réponse impulsionnelle: \newline
La sortie d'un système linéaire invariant est égale au produit de convolution de l'entrée par la réponse impulsionnelle $h, \;\;e(t) \longrightarrow s(t) = (e \otimes h)(t)$


\subsection*{Exercice 1}
Pour les questions (1)-(4), l'entrée est la suite $u_n$ et la sortie la suite $v_n$ pour $n \in \mathbb{Z}$. \newline
Pour les questions (5) et (6) l'entrée est une fonction $f \in \mathrm{L}^1 \cap \mathrm{L}^2$ définie sur $\mathbb{R}$ et la sortie est une fonction $g(x): \; \mathbb{R} \rightarrow \mathbb{R}$
\begin{enumerate}
\item $v_n = u_n - u_{n-1} + 3u_{n+1}$: \newline
La relation est une somme de termes décalés et comporte des produits avec des scalaires, elle est donc linéaire. $\alpha(u_n - u_{n-1} + 3u_{n+1}) \rightarrow \alpha v_n$ \newline
La relation est invariante car $u_{n-\tau} - u_{n-1-\tau} + 3u_{n+1-\tau} \rightarrow v_{n-\tau}$ \newline
La réponse impulsionnelle peut être donnée par identification des termes de la somme du produit de convolution: $h_{-1} = 3, h_0 = 1, h_1 = -1$
\item $v_n = u_{2n}$: \newline
La relation impulsionnelle est un sous échantillonnage qui est linéaire. En effet $\alpha u_{2n} \rightarrow \alpha v_n$. De plus, sous-échantillonner la somme de deux signaux revient à sommer les sous-échantillons de ces signaux.
La relation est non-invariante par translation car $u_{2n-\tau} \centernot\longrightarrow v_{n-\tau} (=u_{2(n-\tau)})$
\item $v_n = \max(u_n, u_{n-1}, u_{n+1})$: \newline
L'opérateur max n'est pas linéaire. Considérons les suites $u_n$ et $u_n'$ définies par: \newline
\begin{itemize}
\item $u_k = u_k' = 0 \;\forall k \in \mathbb{Z} \setminus \{0, 1, 2\}$
\item $u_0 = -1, u_1 = -1, u_2 = -10 $
\item $u_0' = -1, u_1' = 1, u_2' = 10 $
\end{itemize}
Il s'ensuit: 
\begin{equation*}
\begin{split}
 v_n + v_n' &= \max(u_0, u_1, u_2) + \max(u_0', u_1', u_2') \\
			&= \max(-1, -1, -10) + \max(-1, 1, 10) \\
			&= 9 \\
			&\neq \max(-1-1, -1+1, -10 + 10) = 1
\end{split}
\end{equation*}
\item $v_n = u_{n-1}$: \newline
La réponse est linéaire car $\alpha u_{n-1} \rightarrow \alpha v_n$ et invariante par translation \newline car $u_{n-1-\tau} \rightarrow v_{n-\tau}$ \newline
De même que pour la question 1, la réponse impulsionnelle peut être donnée par identification du terme non-nul du produit de convolution: $h_{-1} = 1$
\item $g(x) = \int_{x-\frac{1}{2}}^{x+\frac{1}{2}} f(t) \,\mathrm{d}t$: \newline
La relation est linéaire par linéarité de l'intégration: $$\int_{x-\frac{1}{2}}^{x+\frac{1}{2}} f_1(t) + f_2(t) \,\mathrm{d}t = \int_{x-\frac{1}{2}}^{x+\frac{1}{2}} f_1(t) \,\mathrm{d}t+ \int_{x-\frac{1}{2}}^{x+\frac{1}{2}} f_2(t) \,\mathrm{d}t$$ \newline
La relation est invariante par translation: soit l'entrée $f(t-\tau)$, \newline en posant $t' = t -\tau \implies \mathrm{d}t' = \,\mathrm{d}t $ $$\int_{x-\frac{1}{2}-\tau}^{x+\frac{1}{2}-\tau} f(t')\,\mathrm{d}t' \rightarrow g(x-\tau)$$
Afin de calculer la réponse impulsionnelle, considérons la fonction porte suivante: $\Pi(t) = \mathds{1}\{t \in {[{-\frac{1}{2}}, \frac{1}{2}]}\}$:  $\Pi(t-x) = 1 \;\text{si}\; t-x \in {[{-\frac{1}{2}}, \frac{1}{2}]}$ et 0 sinon.\newline
\begin{equation*}\begin{split}
g(x) = \int_{x-\frac{1}{2}}^{x+\frac{1}{2}} f(t) \,\mathrm{d}t = \int_{-\infty}^{\infty} f(t) \Pi(t-x) \,\mathrm{d}t = (f \otimes \Pi(t-x))(x).
\end{split}\end{equation*}
Donc la réponse impulsionnelle s'écrit $h(t) = \mathds{1}\{t\in {[{-\frac{1}{2}}, \frac{1}{2}]}\}$
\item $g(x) = \max\{f(t), t \in {[x-1, x+1]}\}$: \newline
La réponse n'est pas linéaire car l'opérateur max n'est pas linéaire.
%\begin{equation*}\begin{split}
%\alpha_1 f_1(t) + \alpha_2 f_2(t) & \longrightarrow \max(\alpha_1 f_1(t), \alpha_2 f_2(t)) \\
%								  & \centernot\longrightarrow (\alpha_1 + \alpha_2) \max(f_1(t),f_2(t))
%\end{split}\end{equation*}
%De plus la réponse n'est pas invariante par translation.\newline
%$$f(t-\tau) \centernot\rightarrow \max\{f(t), t \in {[x-1-\tau, x+1-\tau]}\}  $$
\end{enumerate}
\subsection*{Exercice 2}
\begin{enumerate}
\item $u_0 = 1, \; u_n = 0 \; \forall n \neq 0$: \newline
$w_1(n) = (u \otimes v)(n) = (v \otimes u)(n) = v_n = \sqrt{\log(\cos(3n)+2)}$ 
\item $u_0 = 2, u_1 = - \frac{1}{2} , v_0 = 5, v_1 = 3, v_2 = 4$ \newline
L'on donne les termes de $w(n)$ non-nuls ci-dessous:
\begin{itemize}
\item $w_2(0) = u_0 v_0 = 10$
\item $w_2(1) = u_0 v_1 + u_1 v_0 = 3.5$
\item $w_2(2) = u_1 v_1 + u_0 v_2 = 6.5$
\item $w_2(3) = u_1 v_1 + u_1 v_2 = - 3.5$
\end{itemize}
\item $u_{-1} = 2, u_0 = - \frac{1}{2} , v_0 = 5, v_1 = 3, v_2 = 4$ \newline
De même, les termes non-nuls sont:
\begin{itemize}
\item $w_3(-1) = u_0 v_1 + u_{-1} v_0 = 8.5$
\item $w_3(0) = u_0 v_0 + u_{-1} v_1 = 2.5$
\item $w_3(1) = u_{-1} v_2 + u_0 v_1 = 6.5$
\item $w_3(2) = u_0 v_2 = -2$
\end{itemize}
\item $u_{-1} = 2, u_0 = \frac{3}{2}, u_1 = - \frac{1}{2}, v_0 = 5, v_1 = 3, v_2 = 4$ \newline 
En remarquant que les termes des suites $u$ sont la somme des termes des suites des deux question précédentes, la distributivité du produit de convolution donne: 
$w_4(n) = w_2(n) + w_3(n), \; \forall n \in \mathcal{D}(w_2) \cup \mathcal{D}(w_3)$ 
où $\mathcal{D}$ est le domaine de définition de chaque suite.
\item $u_n = (-\frac{1}{2})^n, \; n \in \mathbb{N}, \text{sinon} \; u_n = 0$ et $v_0 = 1, v_1 = \frac{1}{2}$: \newline
En ne gardant que les termes non-nuls du produit de convolution et en factorisant par $(-\frac{1}{2})^{n-1}$
\begin{equation*}\begin{split}
(u \otimes v)(n) &= \sum_{k \in \mathbb{Z}} u_{n-k}v_k \\
		w(n)	 &= (-\frac{1}{2})^n + (\frac{1}{2})(-\frac{1}{2})^{n-1} \\
				 &= (-\frac{1}{2})^{n-1}(\frac{1}{2}-\frac{1}{2}) \\
				 &= 0
\end{split}\end{equation*}
\end{enumerate}

\section*{Résultats de rotation}
L'image de gauche est Lena originale puis chaque image est le résultat d'une rotation d'angle $\theta = \frac{2\pi}{5}$.

\begin{figure}[h]
	\includegraphics[width=0.15\textwidth]{lena.png}
	\includegraphics[width=0.15\textwidth]{rotation_2pi_1.png}
	\includegraphics[width=0.15\textwidth]{rotation_2pi_2.png}
	\includegraphics[width=0.15\textwidth]{rotation_2pi_3.png}
	\includegraphics[width=0.15\textwidth]{rotation_2pi_4.png}
	\includegraphics[width=0.15\textwidth]{rotation_2pi_5.png}
  \caption{Image originale, images après rotations}
\end{figure}


\newpage
\begin{thebibliography}{2}

\bibitem{unser95}
  Unser, Michael, Philippe Thevenaz, and Leonid Yaroslavsky. 
  "Convolution-based interpolation for fast, high-quality rotation of images." 
  IEEE Transactions on Image Processing 4.10 (1995): 1371-1381.
\bibitem{larkin97}  
  Larkin, Kieran G., Michael A. Oldfield, and Hanno Klemm. 
  "Fast Fourier method for the accurate rotation of sampled images." 
  Optics communications 139.1 (1997): 99-106.
\bibitem{paeth86}
	Paeth, Alan W. 
	"A fast algorithm for general raster rotation." 
	Graphics Interface. Vol. 86. 1986.

\end{thebibliography}

\end{document}
