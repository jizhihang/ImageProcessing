\documentclass{article}
% ----- Preamble
\usepackage[utf8]{inputenc} % police encodee en latin1=iso8859-1=Windows Latin 1 %
\usepackage[french]{babel} % police fr %
\usepackage{hyperref} % pour les references %
\usepackage{amsmath} % pour les formules de maths %
\usepackage{amssymb} % pour les symboles maths %
\usepackage{amsthm} % pour la mise en forme des theoremes %
\usepackage{aeguill} % pour les guillemets et accents francais %
\usepackage{listings} % pour les listings de code %
\usepackage{helvet} % police helvetica %
\usepackage{graphicx}
\usepackage{centernot}
\usepackage{dsfont}
% modification des dimensions de la page et de son centrage %
\topmargin 0.0cm
\oddsidemargin 0.2cm
\textwidth 16cm 
\textheight 21cm
\footskip 0.0cm

\title{Traitement d'Image et du Signal - TP4 \\ Transformée en Z}
\author{Laurent Cetinsoy, Karim Kouki, Aris Tritas }
\date{\today}

\begin{document}
\maketitle
\section*{Introduction}
\section{$H(z) = 1 - \frac{1}{2}z^{-1}$}
\subsection*{Réponse impulsionnelle}
$H(z) = \displaystyle\sum_{n \in \mathbb{Z}}{(h_n  \: z^{-1})^{n}}$ donc par identification:
$h_0 = 1, h_1 = -\frac{1}{2}z^{-1}$
\subsection*{Equation de récurrence}
$y_n = \displaystyle\sum_{k \in \mathbb{Z}}{h_n  \: x_{n-k}}$ donc par identification:
$y_n = x_n - \frac{1}{2}x_{n-1}$
\subsection*{Causalité}
La suite $h$ est \textbf{causale}: $h_n = 0 \;,\; \forall n < 0$
\section{$H(z) = 1 +2z^{-1} +3z^{-2}$}
\subsection*{Réponse impulsionnelle}
$h_0 = 1, h_1 = 2z^{-1}, h_2 = 3z^{-2}$
\subsection*{Equation de récurrence}
$y_n = x_n +2x_{n-1} +3x_{n-2}$
\subsection*{Causalité}
La suite $h$ est \textbf{causale}
\section{$H(z) = \frac{1}{1 - \frac{1}{2}z^{-1}}$}
\subsection*{Réponse impulsionnelle}

$H(z) = \frac{1}{1 - \frac{1}{2}z^{-1}} = \displaystyle\sum_{n \in \mathbb{N}}{(\frac{1}{2}z^{-1})^{n}} = \displaystyle\sum_{n \in \mathbb{N}}{(\frac{1}{2})^n z^{-n}}$ (que l'on peut écrire car $ |\frac{1}{2}| < 1)$ \\ $\implies h_n = (\frac{1}{2})^n \;,\; \forall n \geq 0$
\subsection*{Equation de récurrence}
$y_n - \frac{1}{2}y_{n-1}= x_n$
\subsection*{Causalité}
La suite est \textbf{causale}
\section{$H(z) = \frac{1}{1 - 2z^{-1}}$}
\subsection*{Réponse impulsionnelle}
\begin{equation*}\begin{split} 
H(z) &= \frac{1}{1 - 2z^{-1}} = \frac{1}{z^{-1}(z-2)} = \frac{z}{2(\frac{1}{2}z-1)} = -\frac{z}{2} \frac{1}{1-\frac{1}{2}z} \\
&= -\frac{z}{2} \displaystyle\sum_{n=0}^\infty{(\frac{1}{2}z)^{n}} \\
&= -\displaystyle\sum_{n=0}^\infty{(\frac{1}{2}z)^{n+1}} \\
&= -\displaystyle\sum_{m=1}^\infty{(\frac{1}{2}z)^{m}} \hspace{20pt} m = n+1\\
&= -\displaystyle\sum_{k=-\infty}^{-1}{(\frac{1}{2}z)^{-k}} \hspace{8pt} k = -m
\end{split}\end{equation*}
Donc $h_n = -(\frac{1}{2})^n \;,\; \forall n < 0$
\subsection*{Equation de récurrence}
\subsection*{Causalité}
La suite est \textbf{anti-causale}
\end{document}
