\documentclass{article}
% ----- Preamble
\usepackage[utf8]{inputenc} % police encodee en latin1=iso8859-1=Windows Latin 1 %
\usepackage[french]{babel} % police fr %
\usepackage{hyperref} % pour les references %
\usepackage{amsmath} % pour les formules de maths %
\usepackage{amssymb} % pour les symboles maths %
\usepackage{amsthm} % pour la mise en forme des theoremes %
\usepackage{aeguill} % pour les guillemets et accents francais %
\usepackage{listings} % pour les listings de code %
\usepackage{helvet} % police helvetica %
\usepackage{graphicx}
\usepackage{dsfont}
\usepackage{subcaption}
\usepackage{caption}
% modification des dimensions de la page et de son centrage %
\topmargin 0.0cm
\oddsidemargin 0.1cm
\textwidth 16cm 
\textheight 22cm
\footskip 0.0cm

\title{Traitement d'Image et du Signal - TP6}
\author{Laurent Cetinsoy, Karim Kouki, Aris Tritas }
\date{\today}

\begin{document}
\maketitle


\section{Déconvolution par division et par filtre de Weiner}

On compare la déconvolution par division de la TFD et à l'aide du filtre de Wiener.

Par ailleurs, l'on déconvolue l'image par division de la TFD d'une gaussienne d'écart type $\sigma$.
\begin{figure}[h]
	\includegraphics[width=0.3\textwidth]{{deconvolution_sigma_3.5}.png}
	\includegraphics[width=0.3\textwidth]{deconvolution_sigma_4.png}
	\includegraphics[width=0.3\textwidth]{{deconvolution_sigma_4.5}.png}

  \caption{La variance de la gaussienne servant à déconvoluer est égale de gauche à droite, \\ $\sigma = 3.5, \;4,\; 4.5$ .}
\end{figure}

Si la deconvolution brutale marche sur une image sans bruit, elle est complétement inopérante sur les images bruitées cf figure 2B.


\begin{figure}[h]
	\includegraphics[width=0.3\textwidth]{deconvolution_sigma_1.png}
	\includegraphics[width=0.3\textwidth]{deconvolution_sigma_1_noise_0_01.png}
	\includegraphics[width=0.3\textwidth]{deconvolution_wiener_sigma_g_3_5_noise_0_01.png}

  \caption{A gauche (A) l'image sans bruit déconvoluée par division. Au milieu la même image bruitée $\sigma = 0.01 $ (B). A droite l'image bruitée déconvoluée par le filtre de Winer (C). }
\end{figure}

Le filtre de Wiener est robuste à l'ajout de bruit.

\begin{figure}[h]
	\includegraphics[width=0.3\textwidth]{deconvolution_sigma_1.png}
	\includegraphics[width=0.3\textwidth]{deconvolution_sigma_1_noise_0_01.png}
	\includegraphics[width=0.3\textwidth]{deconvolution_wiener_sigma_g_3_5_noise_0_01.png}

  \caption{A gauche l'image déconvoluée par un kernel gaussien, sans bruit. A droite l'ajout de bruit de variance 0.01}
\end{figure}

\section{Echange de phase}

Dans cette partie on s'intéresse à l'information portée par la phase et le module de la TFD d'une image. 


\begin{figure}[h]
	\includegraphics[width=0.3\textwidth]{phase_swapping.jpg}
	\includegraphics[width=0.3\textwidth]{phase_swapping_in_random.jpg}
	\includegraphics[width=0.3\textwidth]{module_swapping.jpg}

  \caption{A gauche la phase de lena qui a été mélangée au module de barbara. Au milieu, la phase de lena projetée sur une image générée par une gaussienne. A droite le module de Lena et Barbara ont été échangés.}
\end{figure}

On constate qu'une grande partie de l'information est portée par la phase. Inversement on voit que si on transfère le module de l'image de Lena dans Barbara, on voit bien barbara (qui est quand même moins sexy). 

Néanmoins la phase seule ne suffit pas (cf figure.2) : si on transporte la phase sur une image uniforme, l'image reste inchangée après transport de phase. 

\begin{figure}[h]

	\includegraphics[width=0.3\textwidth]{phase_swapping_in_grey_image.jpg}

  \caption{Phase de lena exportée dans une image constante (gris total)}
\end{figure}


On voit que le module ne porte que peu d'information mais a quand même besoin d'être présent. 

\newpage

\section{Ré-haussement des couleurs}


\begin{figure}[h]

	\includegraphics[width=0.3\textwidth]{edouart-manet-berthe-morisot.jpg}
	\includegraphics[width=0.3\textwidth]{edouart-manet-berthe-morisot-boosted.jpg}
	\includegraphics[width=0.3\textwidth]{edouart-manet-berthe-morisot-deboosted.jpg}
  \caption{A gauche image originale. Au milieu, H augmentée de 150 \%. A droite diminuée de 50 \%}
\end{figure}

\newpage

\section{Changement de contraste}

\subsection*{Question 1}

\begin{figure}[h]

	\includegraphics[width=0.5\textwidth]{hist_orig.png}
	\includegraphics[width=0.5\textwidth]{hist_scaled.png}
		
	\includegraphics[width=0.5\textwidth]{lena.png}
	\includegraphics[width=0.5\textwidth]{lena_hist_scaled.png}
		
	\caption{A gauche l'image et l'histogramme originaux. A droite après changement de contraste affine}
	
\end{figure}

On constate que l'image de droite a un contraste renforcée. On voit que l'histogramme est davantage étiré. 

\subsection*{Question 2}

Dans le cas où il existe dans l'image un pixel qui est nul et un pixel d'intensité maximal (1). Alors la transformation affine n'a aucun effet : 


$$min(I(i,j) = 0$$
$$max(I(i,j)) = 1$$
	
$$I'(i,j) = \frac{1}{min-max} * (I(i,j) - min) = I(i,j)$$


\begin{figure}[h]
		
	\caption{A gauche l'image et l'histogramme originaux. A droite après changement de contraste affine}
	
\end{figure}


On peut remplacer $min(I(i,j)$ par $min(I(i,j)) + \epsilon$ et $max$ par $max(I(i,j)$ par $max(I(i,j) - \epsilon$ (à Tester)


\section{Filtre médian}

\section{Travail écrit}
On rapelle que la transformée de Fourier a l'expression suivante:
$$ f_K : \mathbb{R} \ni x \rightarrow \frac{\mathds{1}_{[0, K]}(x)}{K}, K \in \mathbb{R}_*^+ $$
$$ \hat{f}(\xi) = \int_\mathbb{R} f(x) e^{-ix\xi} \:\mathrm{d}x$$

\subsection{Convolution}

\begin{equation*}\begin{split}
\hat{f_K}(\xi) 
&= \int_\mathbb{R} \frac{\mathds{1}_{[0, K]}(x)}{K} e^{-ix\xi} \:\mathrm{d}x \;=\; \int_0^{K} \frac{1}{K}e^{-ix\xi}\:\mathrm{d} x \\
&= -\frac{1}{i \xi K} e^{\frac{-i \xi K}{2}}\bigg [ e^{\frac{-i \xi K}{2}} - e^{\frac{i \xi K}{2}} \bigg] \\
&= \frac{2}{\xi K} e^{\frac{-i \xi K}{2}} \sin(\frac{\xi K}{2\pi})\\
&= e^{\frac{-i \xi K}{2}} \text{sinc}(\frac{\xi K}{2\pi})
\end{split}\end{equation*}

\subsection{Flou de bougé}
\begin{enumerate}
\item Si K est strictement positif, le dénominateur de la transformée de Fourier calculée précédemment ne s'annule pas. La convolution par $f_K$ est donc inversible.

On calcule ainsi la convolution de $f = u$ et $g =f_{v \Delta t} \times \Delta t $. Ainsi,
\begin{equation*}\begin{split}
(f * g)(x) = \int_\mathbb{R} u(x-z)\frac{\mathds{1}_{[0, v \Delta t]}(x)}{v \Delta t} \Delta t \:\mathrm{d}z \; = \;   \int_{0}^{v \Delta t} \frac{u(x-z)}{v} \:\mathrm{d}z
\end{split}\end{equation*}

En posant le changement de variable affine: $ z=v t $, on obtient les bonnes bornes d'intégration et la formule souhaitée:
$$(f * g)(x) = \int_{0}^{\Delta t} u(x-vt)\:\mathrm{d}t$$
\item Le flou perçu dépend de $v$:

\begin{itemize}
\item Si pendant le temps $\Delta t$, la distance $v \Delta t$ parcourue par l'objet observé est inférieure à la résolution du capteur, l'image observée ne sera pas forcément perçue plus floue.
\item Inversement, si pendant le temps $\Delta t$, la distance $v \Delta t$ est suffisamment importante, plus les zones réceptrices du capteur vont recevoir des photons d'un objet qui a bougé d'autant plus avec l'augmentation de $\Delta t$. Dans ce cas de figure, l'objet est perçu plus flou.
\end{itemize}

\item Pour simuler un flou de bougé, on peut utiliser l'algorithme:
\begin{itemize}
\item On choisit $v \; et \; \Delta t$ 
\item On calcule la TFD de l'image et celle de la fonction porte
\item On réalise le produit de convolution par multiplication des TFD de l'image et de la fonction porte
\item Finalement on prend la TFD inverse afin de pouvoir visualiser l'image.
\end{itemize}

\end{enumerate}
\end{document}